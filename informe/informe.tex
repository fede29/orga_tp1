\documentclass[11pt,a4paper]{article}

%Paquetes importados
\usepackage[utf8]{inputenc}
\usepackage[spanish]{babel}
\usepackage{graphicx}
\usepackage{amssymb}
\usepackage{amsmath}
\usepackage{float}
\usepackage{listings}
\usepackage[rgb,svgnames,table]{xcolor}

\addto\captionsspanish{
	\renewcommand\tablename{Tabla}
	\renewcommand\listtablename{Lista de tablas}
	\renewcommand\lstlistingname{Código}
	\renewcommand\lstlistlistingname{Lista de código}
}

\lstset{ % Defino el formato de bloques de código fuente
	inputencoding=utf8, % Indico la codificación de los archivos de entrada
	extendedchars=true, % Extiendo los caracteres
	% Escapeo caracteres especiales
	literate={á}{{\'a}}1 {é}{{\'e}}1 {í}{{\'i}}1 {ó}{{\'o}}1 {ú}{{\'u}}1 {ñ}{{\~n}}1,
	showtabs=false, % Indica si se muestran los tabs
	tabsize=2, % Indica la cantidad de espacios que ocupa un tab
	showspaces=false, % Indica si muestra los espacios
	showstringspaces=false, % Indica si muestra los espacios dentro de strings
	numbers=left, % Posición en que se muestran los números de línea
	numberstyle=\tiny\color{gray}, % Estilo de los números de línea
	breaklines=true, % Se parten las líneas que exceden del ancho del documento
	frame=single, % Formato del marco del entorno
	backgroundcolor=\color{gray!5}, % Color de fondo
	basicstyle=\ttfamily\footnotesize, % Estilo de base (familia, tamaño, color)
	keywordstyle=\color{DarkBlue}, % Estilo de las palabras reservadas
	stringstyle=\color{red}, % Estilo de los strings
	commentstyle=\color{DarkGreen}, % Estilo de los comentarios
	language=Octave, % Espeficica el lenguaje a usar
	otherkeywords={std,cout} % Agrego palabras reservadas que no se resaltan
}

\begin{document}
\title{TP Analisis de la Informacion}
\author{Federico Rodriguez Longhi}		
\date{\today}

\begin{titlepage}
	
	\begin{figure}[H]
		\raggedright
		\includegraphics[scale=0.25]{logo_fiuba2}
		\hfill
		\raggedleft
		\includegraphics[scale=0.2]{logo_uba}
	\end{figure}
	\rule{\textwidth}{1pt}\par % Thick horizontal line
	\vspace{2pt}\vspace{-\baselineskip} % Whitespace between lines
	\rule{\textwidth}{0.4pt}\par % Thin horizontal line
	
	\vspace{0.05\textheight} % Whitespace between the top lines and title
	\centering % Center all text
	{\Huge UBA FACULTAD DE INGENIERÍA}\\[0.5\baselineskip]
	{\Large 66.20 Organización de Computadoras}\\[0.5\baselineskip]
	{\Huge Trabajo Practico}\\[0.75\baselineskip]
	{\Large 2$^{do}$ Cuatrimestre 2016}\\[0.5\baselineskip]
	\vspace{0.2\textheight}
	
	\begin{table}[H]
		\begin{flushleft}
		{\Large Grupo 1}\\
		\vspace{0.01\textheight}
		\textbf{Integrantes:}\\
		\vspace{0.01\textheight}
		\begin{tabular}{l r}
			Federico Rodriguez  & 93336\\
			\hspace{0.05\textwidth}fede.longhi@hotmail.com&\\
			Ezequiel Dufau & 91985\\
			\hspace{0.05\textwidth}fede.longhi@hotmail.com&\\
			Pablo Ascarza & 89711\\
			\hspace{0.05\textwidth}fede.longhi@hotmail.com&\\
		\end{tabular}
		\end{flushleft}
	\end{table}
	
	\vspace{0.05\textheight}
	\vspace{2pt}
	\vfill
	\rule{\textwidth}{1pt}\par % Thick horizontal line
	\vspace{2pt}\vspace{-\baselineskip} % Whitespace between lines
	\rule{\textwidth}{0.4pt}\par % Thin horizontal line
	
\end{titlepage}

\tableofcontents
\newpage

\section{Enunciado}

\section{Introducción}
El presente trabajo tiene como objetivo familiarizarse con el conjunto de instrucciones \emph{MIPS-32} y el concepto de ABI. Para el cumplimiento de este objetivo se desarrolló un programa que simula el \emph{``Juego de la Vida'' de Conway} según lo detallado en el enunciado.

La implementación se realizo en el lenguaje de programacion C. Además se desarrollo una porción en assembler \emph{MIPS-32} que luego será detallada.

El programa fue desarrollado para correr sobre una plataforma \emph{NetBSD / MIPS-32} mediante el emulador \emph{GXEmul}.



\section{Utilización}
El programa fue implementado para que cumpliera con los requisitos pedidos por el tp.
En las siguientes secciones se detallarán los diferentes aspectos para la ejecución del programa.

\subsection{Compilación y Ejecución}
\begin{enumerate}
	\item Descargar el archivo fuente ``conway.c"
	\item Compilar el archivo (por ejemplo con gcc:\\
	\texttt{gcc -Wall -c  ``conway.c''\\
	gcc -Wall -o ``conway'' ``conway.c''})
	\item Ejecutar el programa con: \texttt{./conway i M N input [-o output]}
\end{enumerate}

\subsection{Documentación de Parámetros}
\begin{itemize}
	\item \texttt{i} es la cantidad de simulaciones que queremos realizar.
	\item \texttt{M} y \texttt{N} especifican las dimensiones de la matriz sobre la cual queremos simular.
	\item \texttt{input} es el nombre del archivo que contiene las coordenadas de las celdas vivas e identifica el estado inicial de la matriz.
	\item \texttt{-o} es un parámetro opcional que especifica que se utilizará el nombre \texttt{output} como prefijo de los nombres de los archivos pbm generados. En caso de no existir este parámetro tomara como prefijo \texttt{input}.
	\item \texttt{-V} o \texttt{--version} muestra la versión del programa.
	\item \texttt{-h} o \texttt{--help} muestra la ayuda.
\end{itemize}

\subsection{Documentación de Errores}
A continuación se detallan los errores y su significado:
\begin{itemize}
	\item \textbf{Actions Count must be a positive integer:} El primer parámetro tiene que ser un entero positivo.
	\item \textbf{Number of rows must be a positive integer:} El número de filas tiene que ser un entero positivo.
	\item \textbf{Number of columns must be a positive integer:}El número de columnas tiene que ser un entero positivo.
	\item \textbf{Invalid parameter:} Para el caso de \texttt{-V}, \texttt{-h} y \texttt{-o}. El parámetro no coincide con estos valores (o sus equivalentes).
	\item \textbf{Wrong number of parameters:} Hay parámetros de mas o de menos (se pasó un número de parámetros distinto a 1 o 6).
	\item \textbf{Error while opening input file:} No se pudo abrir el archivo de entrada.
	\item \textbf{Error while opening output file: [nombre\_de\_archivo]:} No se pudo abrir el archivo de salida con el nombre \emph{nombre\_de\_archivo}
\end{itemize}

\subsection{Algunas Aclaraciones}
\begin{itemize}
	\item Las imágenes pbm generadas se guardan en la carpeta imágenes.
	\item Todos los errores se imprimen directamente a \texttt{stderr}.
	
\end{itemize}

\subsection{Ejemplos de Uso}

Para ver la documentación:\\

\texttt{./conway -h}\\

Para ver la informacion sobre la version:\\

\texttt{./conway -V}\\

Para generar un tablero de $100 \times 50$ a partir del archivo \texttt{glider} y realizar 20 iteraciones:\\

\texttt{./conway 20 100 50 glider}\\

Los archivos pbm generados por el comando anterior seran nombrados de la forma: ``glider\_N.pbm''.\\

Para generar un tablero de $20 \times 30$ a partir del archivo \texttt{pento}, realizar 10 iteraciones y que los archivos pbm generados tengan como prefijo el nombre \texttt{jorge}:\\

\texttt{./conway 10 20 30 pento -o jorge}\\

Los archivos pbm generados por el comando anterior seran nombrados de la forma: ``jorge\_N.pbm''.

\section{Implementación}
En esta sección se presentar porciones del programa. Para ver el código completo dirigirse al apéndice A.

Para la implementación se diseño una función en C según la documentación del enunciado que corresponde a:\\
\texttt{unsigned int vecinos(unsigned char *a,\\
	unsigned int i, unsigned int j,\\
	unsigned int M, unsigned int N);}\\

Tenemos que considerar un detalle sobre el tratamiento de la matriz para entender el algoritmo. En la figura siguiente se muestra la conversión de la matriz a array que utilizamos.

\begin{equation*}
\begin{bmatrix}
0,0 & 0,1 & 0,2 \\
1,0 & 1,1 & 1,2 \\
2,0 & 2,1 & 2,2
\end{bmatrix}
\rightarrow 
\begin{bmatrix}
\underset{0}{0,0} & \underset{1}{0,1} & \underset{2}{0,2} & \underset{3}{1,0} & \underset{4}{1,1} & \underset{5}{1,2} & \underset{6}{2,0} & \underset{7}{2,1} & \underset{8}{2,2}
\end{bmatrix}
\end{equation*}

A continuación se muestra el algoritmo implementado en C:

\begin{lstlisting}[caption={Código de la funcion vecinos},label={lst:codigoc}]
unsigned int vecinos(unsigned char *a, unsigned int i, unsigned int j, unsigned int M, unsigned int N){
	unsigned int vecinos = 0;
	int x,y;
	for (x = -1; x<2; x++){
		for (y = -1; y<2; y++){
			if (x+i>=0 && x+i<N && y+j>=0 && y+j<M){
				if ((!(x==0&&y==0)) && a[N*(x+i) + (y+j)] == '1'){
					vecinos++;
				}
			}
		}
	}
	return vecinos;
}
\end{lstlisting}

\section{Documentación Diseño e Implementación}
\subsection{Diagrama del Stack de Vecinos}

\section{Corridas de Prueba}

\section{Conclusiones}


\appendix
\part*{Apéndice}
\section{Código Fuente}



\end{document}
