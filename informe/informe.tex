\documentclass[11pt,a4paper]{article}

%Paquetes importados
\usepackage[utf8]{inputenc}
\usepackage[spanish]{babel}
\usepackage{graphicx}
\usepackage{amssymb}
\usepackage{amsmath}
\usepackage{float}
\usepackage{listings}
\usepackage[rgb,svgnames,table]{xcolor}

\addto\captionsspanish{
	\renewcommand\tablename{Tabla}
	\renewcommand\listtablename{Lista de tablas}
	\renewcommand\lstlistingname{Código}
	\renewcommand\lstlistlistingname{Lista de código}
}

\lstset{ % Defino el formato de bloques de código fuente
	inputencoding=utf8, % Indico la codificación de los archivos de entrada
	extendedchars=true, % Extiendo los caracteres
	% Escapeo caracteres especiales
	literate={á}{{\'a}}1 {é}{{\'e}}1 {í}{{\'i}}1 {ó}{{\'o}}1 {ú}{{\'u}}1 {ñ}{{\~n}}1,
	showtabs=false, % Indica si se muestran los tabs
	tabsize=2, % Indica la cantidad de espacios que ocupa un tab
	showspaces=false, % Indica si muestra los espacios
	showstringspaces=false, % Indica si muestra los espacios dentro de strings
	numbers=left, % Posición en que se muestran los números de línea
	numberstyle=\tiny\color{gray}, % Estilo de los números de línea
	breaklines=true, % Se parten las líneas que exceden del ancho del documento
	frame=single, % Formato del marco del entorno
	backgroundcolor=\color{gray!5}, % Color de fondo
	basicstyle=\ttfamily\footnotesize, % Estilo de base (familia, tamaño, color)
	keywordstyle=\color{DarkBlue}, % Estilo de las palabras reservadas
	stringstyle=\color{red}, % Estilo de los strings
	commentstyle=\color{DarkGreen}, % Estilo de los comentarios
	language=Octave, % Espeficica el lenguaje a usar
	otherkeywords={std,cout} % Agrego palabras reservadas que no se resaltan
}

\begin{document}
\title{TP Organizacion de Computadoras}
\author{Federico Rodriguez Longhi}		
\date{\today}

\begin{titlepage}
	
	\begin{figure}[H]
		\raggedright
		\includegraphics[scale=0.25]{logo_fiuba2}
		\hfill
		\raggedleft
		\includegraphics[scale=0.2]{logo_uba}
	\end{figure}
	\rule{\textwidth}{1pt}\par % Thick horizontal line
	\vspace{2pt}\vspace{-\baselineskip} % Whitespace between lines
	\rule{\textwidth}{0.4pt}\par % Thin horizontal line
	
	\vspace{0.05\textheight} % Whitespace between the top lines and title
	\centering % Center all text
	{\Huge UBA FACULTAD DE INGENIERÍA}\\[0.5\baselineskip]
	{\Large 66.20 Organización de Computadoras}\\[0.5\baselineskip]
	{\Huge Trabajo Practico}\\[0.75\baselineskip]
	{\Large 2$^{do}$ Cuatrimestre 2016}\\[0.5\baselineskip]
	\vspace{0.2\textheight}
	
	\begin{table}[H]
		\begin{flushleft}
		{\Large Grupo 1}\\
		\vspace{0.01\textheight}
		\textbf{Integrantes:}\\
		\vspace{0.01\textheight}
		\begin{tabular}{l r}
			Federico Rodriguez  & 93336\\
			\hspace{0.05\textwidth}fede.longhi@hotmail.com&\\
			Ezequiel Dufau & 91985\\
			\hspace{0.05\textwidth}fede.longhi@hotmail.com&\\
			Pablo Ascarza & 89711\\
			\hspace{0.05\textwidth}fede.longhi@hotmail.com&\\
		\end{tabular}
		\end{flushleft}
	\end{table}
	
	\vspace{0.05\textheight}
	\vspace{2pt}
	\vfill
	\rule{\textwidth}{1pt}\par % Thick horizontal line
	\vspace{2pt}\vspace{-\baselineskip} % Whitespace between lines
	\rule{\textwidth}{0.4pt}\par % Thin horizontal line
	
\end{titlepage}

\tableofcontents
\newpage

\section{Enunciado}

\section{Introducción}
El presente trabajo tiene como objetivo familiarizarse con el conjunto de instrucciones \emph{MIPS-32} y el concepto de ABI. Para el cumplimiento de este objetivo se desarrolló un programa que simula el \emph{``Juego de la Vida'' de Conway} según lo detallado en el enunciado.

La implementación se realizo en el lenguaje de programacion C. Además se desarrollo una porción en assembler \emph{MIPS-32} que luego será detallada.

El programa fue desarrollado para correr sobre una plataforma \emph{NetBSD / MIPS-32} mediante el emulador \emph{GXEmul}.

\section{Utilización}
El programa fue implementado para que cumpliera con los requisitos pedidos por el tp.
En las siguientes secciones se detallarán los diferentes aspectos para la ejecución del programa.

\subsection{Compilación y Ejecución}
\begin{enumerate}
	\item Descargar el archivo Conway.zip desde el repositorio\footnote{https://github.com/fede29/orga\_tp1}.\\
	\item Descomprimir en el directorio que desee.
	\item Desde el directorio donde se descomprimieron los archivos, ejecutar \texttt{make} para generar el código enteramente en C o \texttt{make mips} para generar el código con la función vecinos en Mips.
	\item Ejecutar el programa con: \texttt{./conway i M N input [-o output]}
\end{enumerate}

\subsection{Documentación de Parámetros}
\begin{itemize}
	\item \texttt{i} es la cantidad de simulaciones que queremos realizar.
	\item \texttt{M} y \texttt{N} especifican las dimensiones de la matriz sobre la cual queremos simular.
	\item \texttt{input} es el nombre del archivo que contiene las coordenadas de las celdas vivas e identifica el estado inicial de la matriz.
	\item \texttt{-o} es un parámetro opcional que especifica que se utilizará el nombre \texttt{output} como prefijo de los nombres de los archivos pbm generados. En caso de no existir este parámetro tomara como prefijo \texttt{input}.
	\item \texttt{-V} o \texttt{--version} muestra la versión del programa.
	\item \texttt{-h} o \texttt{--help} muestra la ayuda.
\end{itemize}

\subsection{Documentación de Errores}
A continuación se detallan los errores y su significado:
\begin{itemize}
	\item \textbf{Actions Count must be a positive integer:} El primer parámetro tiene que ser un entero positivo.
	\item \textbf{Number of rows must be a positive integer:} El número de filas tiene que ser un entero positivo.
	\item \textbf{Number of columns must be a positive integer:}El número de columnas tiene que ser un entero positivo.
	\item \textbf{Invalid parameter:} Para el caso de \texttt{-V}, \texttt{-h} y \texttt{-o}. El parámetro no coincide con estos valores (o sus equivalentes).
	\item \textbf{Wrong number of parameters:} Hay parámetros de mas o de menos (se pasó un número de parámetros distinto a 1 o 6).
	\item \textbf{Error while opening input file:} No se pudo abrir el archivo de entrada.
	\item \textbf{Error while opening output file: [nombre\_de\_archivo]:} No se pudo abrir el archivo de salida con el nombre \emph{nombre\_de\_archivo}
\end{itemize}

\subsection{Algunas Aclaraciones}
\begin{itemize}
	\item Las imágenes pbm generadas se guardan en la carpeta imágenes.
	\item Todos los errores se imprimen directamente a \texttt{stderr}.
	
\end{itemize}

\subsection{Ejemplos de Uso}

Para ver la documentación:\\

\texttt{./conway -h}\\

Para ver la informacion sobre la version:\\

\texttt{./conway -V}\\

Para generar un tablero de $100 \times 50$ a partir del archivo \texttt{glider} y realizar 20 iteraciones:\\

\texttt{./conway 20 100 50 glider}\\

Los archivos pbm generados por el comando anterior seran nombrados de la forma: ``glider\_N.pbm''.\\

Para generar un tablero de $20 \times 30$ a partir del archivo \texttt{pento}, realizar 10 iteraciones y que los archivos pbm generados tengan como prefijo el nombre \texttt{jorge}:\\

\texttt{./conway 10 20 30 pento -o jorge}\\

Los archivos pbm generados por el comando anterior seran nombrados de la forma: ``jorge\_N.pbm''.

\section{Implementación}
En esta sección se presentar porciones del programa. Para ver el código completo dirigirse al apéndice A.

La implementación se hizo completamente en C. Luego se programo en MIPS la función vecinos. A continuación se detallan las dos implementaciones.

\subsection{Implementación en C}

Para la implementación se diseño una función en C según la documentación del enunciado que corresponde a:\\
\texttt{unsigned int vecinos(unsigned char *a,\\
	unsigned int i, unsigned int j,\\
	unsigned int M, unsigned int N);}\\

Tenemos que considerar un detalle sobre el tratamiento de la matriz para entender el algoritmo. En la figura siguiente se muestra la conversión de la matriz a array que utilizamos.

\begin{equation*}
\begin{bmatrix}
0,0 & 0,1 & 0,2 \\
1,0 & 1,1 & 1,2 \\
2,0 & 2,1 & 2,2
\end{bmatrix}
\rightarrow 
\begin{bmatrix}
\underset{0}{0,0} & \underset{1}{0,1} & \underset{2}{0,2} & \underset{3}{1,0} & \underset{4}{1,1} & \underset{5}{1,2} & \underset{6}{2,0} & \underset{7}{2,1} & \underset{8}{2,2}
\end{bmatrix}
\end{equation*}

A continuación se muestra el algoritmo implementado en C:

\begin{lstlisting}[caption={Código de la funcion vecinos},label={lst:codigoc}]
unsigned int vecinos(unsigned char *a, unsigned int i, unsigned int j, unsigned int M, unsigned int N){
	unsigned int vecinos = 0;
	int x,y;
	for (x = -1; x<2; x++){
		for (y = -1; y<2; y++){
			if (x+i>=0 && x+i<N && y+j>=0 && y+j<M){
				if ((!(x==0&&y==0)) && a[N*(x+i) + (y+j)] == '1'){
					vecinos++;
				}
			}
		}
	}
	return vecinos;
}
\end{lstlisting}

\subsection{Implementación en MIPS}

Para ver el código fuente dirigirse a la sección A.2.

\subsubsection{Diagrama del Stack de Vecinos}

\section{Corridas de Prueba}

\subsection{Funcionamiento}
Realizamos varias corridas de prueba con los archivos proporcionados (pento, glider y sapo) para verificar el correcto funcionamiento del programa. Mostramos a continuación algunas salidas de las corridas:

\texttt{./conway 10 10 10 glider -s}

\begin{lstlisting}
fedelonghi@fedelonghi-dell $ ./conway 5 10 10 glider -s
starting action
Simulation number: 0

..........
..........
..........
....O.....
.....O....
...OOO....
..........
..........
..........
..........
Simulation number: 1

..........
..........
..........
..........
...O.O....
....OO....
....O.....
..........
..........
..........
Simulation number: 2

..........
..........
..........
..........
.....O....
...O.O....
....OO....
..........
..........
..........
Simulation number: 3

..........
..........
..........
..........
....O.....
.....OO...
....OO....
..........
..........
..........
Simulation number: 4

..........
..........
..........
..........
.....O....
......O...
....OOO...
..........
..........
..........


\end{lstlisting}

\texttt{./test 10 10 10 glider -o archivo\_salida}

\begin{lstlisting}
fedelonghi@fedelonghi-dell $ ./test 10 10 10 glider -o archivo_salida
starting action
\end{lstlisting}

Los archivos .pbm fueron generados en el directorio imagenes.

\texttt{./test 3 10 10 sapo -s}

\begin{lstlisting}
fedelonghi@fedelonghi-dell $ ./test 3 10 10 sapo -s
starting action
Simulation number: 0

..........
..........
..........
..........
....OOO...
...OOO....
..........
..........
..........
..........
Simulation number: 1

..........
..........
..........
.....O....
...O..O...
...O..O...
....O.....
..........
..........
..........
Simulation number: 2

..........
..........
..........
..........
....OOO...
...OOO....
..........
..........
..........
..........

\end{lstlisting}

\subsection{Resultados}
Además medimos los tiempos de ejecucion para el programa compilado enteramente en c y el compilado con la función vecinos en mips. A continuación se muestran los resultados\footnote{Todos los tiempos están medidos en segundos.}:

\begin{table}[H]
	\centering
	\begin{tabular}{|l|c|c|c|c|c|c|}
		\hline
		\textbf{input}&\textbf{it}&\textbf{M}&\textbf{N} &\textbf{time 1}&\textbf{time 2}&\textbf{time 3}\\
		\hline
		pento &	10	&10	&10	&0,098	&0,117	&0,102\\
		      & 100	&10	&10	&0,848	&0,836	&0,816\\
		      & 10	&20	&20	&0,832	&0,781	&0,750\\
		\hline
		glider& 10	&10	&10	&0,102	&0,102	&0,102\\
		      & 100	&10	&10	&0,836	&0,836	&0,785\\
		      & 10	&20	&20	&0,766	&0,77	&0,746\\
		\hline
		sapo  &	10	&10	&10	&0,102	&0,117	&0,102\\
		      & 100	&10	&10	&0,848	&0,836	&0,816\\
		      & 10	&20	&20	&0,73	&0,766	&0,750\\
		\hline
	\end{tabular}
	\caption{Tiempo de ejecución del programa compilado enteramente C}
	\label{tab:t_c}
\end{table}

\begin{table}[H]
	\centering
	\begin{tabular}{|l|c|c|c|c|c|c|}
		\hline
		\textbf{input}&\textbf{it}&\textbf{M}&\textbf{N} &\textbf{time 1}&\textbf{time 2}&\textbf{time 3}\\
		\hline
		pento	&10	 &10	&10 &0,098	&0,098	&0,102\\
				&100 &10	&10 &0,797	&0,801	&0,77\\
				&10	 &20	&20 &0,746	&0,77	&0,801\\
				\hline
		glider	&10	 &10	&10 &0,102	&0,098	&0,082\\
				&100 &10	&10	&0,816	&0,801	&0,781\\
				&10	 &20	&20	&0,77	&0,73	&0,75\\
				\hline
		sapo	&10	 &10	&10	&0,098	&0,102	&0,098\\
				&100 &10	&10	&0,801	&0,77	&0,77\\
				&10	 &20	&20	&0,785	&0,766	&0,77\\
		\hline
	\end{tabular}
	\caption{Tiempo de ejecución del programa compilado con la función vecinos en mips}
	\label{tab:t_mips}
\end{table}

Para tener una mejor noción sobre los tiempos a continuación se muestra el promedio. El promedio se calculo sin considerar el tipo de archivo de input, es decir considerando los todos los tiempos pertenecientes a las ejecuciones con los mismos parámetros i M y N.

\begin{table}[H]
	\centering
	\begin{tabular}{|l|c|c|c|c|c|c|}
	\hline
	\textbf{Parámetros}&\textbf{Tiempo (C)}&\textbf{ Tiempo (Mips)}\\
	\hline
	10 10x10 &	0,1048&	0,0975\\
	\hline
	100 10x10&	0,8285&	0,7896\\
	\hline
	10 20x20 &	0,7656&	0,7653\\		
	\hline
	\end{tabular}
	\caption{Promedio de los tiempos de ejecución}
	\label{tab:prom}
\end{table}

\subsection{Análisis de Resultados}
Observando las tablas podemos ver claramente que hay una tendencia a disminuir el tiempo de ejecucion utilizando el programa compilado en mips. Esto se ve especialmente analizando la tabla \ref{tab:prom}. Donde claramente el promedio en todos los casos es menor en mips que en C.

\section{Conclusiones}
Además de familiarizarnos con las herramientas y el set de instrucciones de Mips pudimos observar como programando una función en mips podemos reducir el costo de tiempo en la ejecucion de un programa.

A la vista de los resultados, la ganancia es muy pequeña. Podemos deducir que se trata debido a que el programa en sí consta de otras partes que tienen mucha injerencia en el costo de tiempo del programa (como puede ser la creación de archivos pbm).

Vale aclarar también que la complejidad de programar una función en Mips es mucho mayor que la de programar una función en C. Mas allá del hecho de que estamos aprendiendo a programar en Mips, podemos notar la mayor cantidad de lineas que tiene el código de la función vecinos.s. Esta claro que para lograr un mismo resultado (con funcionalidades no tan pequeñas) en código assembler que en C es necesario programar mucho más.

\appendix
\part*{Apéndice}
\section{Código Fuente}

\subsection{Programa Principal en C}
\begin{lstlisting}[caption={cornway.c},label={lst:codigoc}]
#include <stdlib.h>
#include <stdio.h>
#include <string.h>
#include <stdbool.h>

void writePBM(unsigned char** board, unsigned int dimx, unsigned int dimy, const char* fileName, unsigned int actionNumber)
{
int i, j;
int actionNumberStringLenght = 1;
if (actionNumber >= 10){
actionNumberStringLenght = 2;
}
char str[actionNumberStringLenght];
sprintf(str, "%d", actionNumber);
char newFileName[50];
newFileName[0]='\0';
strcat(newFileName,"imagenes/");
strcat(newFileName,fileName);
strcat(newFileName,"_");
strcat(newFileName,str);
strcat(newFileName,".pbm");

FILE *fp = fopen(newFileName, "wb");
if (fp==NULL){
fprintf(stderr, "Error while opening output file: %s\n",newFileName);
exit(1);
}

unsigned int amp = 1;
(void) fprintf(fp, "P4\n%d %d\n", dimy*amp, dimx*amp);
for (j = 0; j < dimy; ++j){
for (i = 0; i < dimx; ++i)
{
unsigned int x,y;
unsigned char writeValue = 0;
if ((board[i][j]) == '1'){
writeValue = 1;
}
for (x = i*amp; x < amp*(i+1); x++){
for (y = j*amp; y < amp*(j+1); y++){
fprintf(fp, "%c", writeValue);
}
}
}
}
(void) fclose(fp);
return;
}

unsigned char ** init_board(unsigned int rows, unsigned int cols){
unsigned char** board = (unsigned char**)malloc(cols*sizeof(char*));
unsigned int i,j;
for (i = 0; i<cols; i++){
board[i] = (unsigned char *) malloc(rows*sizeof(char*));
for (j = 0; j<rows; j++){
board[i][j]='0';
}
}
return board;
}

void load_board(unsigned char **board, unsigned int rows, unsigned int cols, FILE *file){
int x,y;
while (fscanf (file, "%i %i", &x, &y)!=EOF){
if (x<cols||y<rows){
board[x][y]='1';
}
}
}

int free_board (unsigned char **board, unsigned int rows, unsigned int cols){
unsigned int i;
for (i = 0; i<cols; i++){
free(board[i]);
}
free(board);
return 0;
}

void print_board(unsigned char **board, unsigned int rows, unsigned int cols){
unsigned int i,j;
for (i = 0; i<cols; i++){
for (j = 0; j<rows; j++){
if (board[i][j] == '0'){
printf(".");
}else{
printf("O");
}
}
printf("\n");
}
}

void print_board_file(unsigned char **board, unsigned int rows, unsigned int cols,int n,const char *filename){
char newFilename [50];
int numLength = 1;
if (n >= 10 && n <= 99){
numLength = 2;
}else if (n >= 100 && n <= 999){
numLength = 3;
}
char str[numLength];
sprintf(str, "%d", n);
newFilename[0] = '\0';
strcat(newFilename, "output/");
strcat(newFilename, filename);
strcat(newFilename, "_");
strcat(newFilename, str);
FILE *file = fopen(newFilename,"w+");
fprintf(file, "Simulacion N: %i\n",n);
unsigned int i,j;
for (i = 0; i<cols; i++){
for (j = 0; j<rows; j++){
if (board[i][j] == '0'){
fprintf(file,".");
}else{
fprintf(file,"O");
}
}
fprintf(file,"\n");
}
fclose(file);
}

unsigned char* board_to_array(unsigned char ** board, unsigned int rows, unsigned int cols){
unsigned char* array = (unsigned char*)malloc(sizeof(char*)*rows*cols);
int x,y;
for (x=0; x<cols; x++){
for (y=0; y<rows; y++){
unsigned int pos = cols*x+y;
array[pos]=board[x][y];
}
}
return array;
}

void copy_board(unsigned char **from, unsigned char **to, unsigned int rows, unsigned int cols){
unsigned int i,j;
for (i = 0; i<cols; i++){
for (j = 0; j<rows; j++){
to[i][j] = from[i][j];
}
}
}

void process_board(unsigned char **board, unsigned int rows, unsigned int cols, unsigned int actionCount,const char* fileName,bool screen){
printf ("starting action\n");
unsigned char **thisBoard = init_board(rows, cols);
copy_board(board,thisBoard,rows, cols);
unsigned int i,x,y;
for (i = 0; i < actionCount; i++){
unsigned char **nextBoard = init_board(rows, cols);
if (screen){
printf("Simulation number: %i\n\n", i);
print_board(thisBoard,rows,cols);
print_board_file(thisBoard,rows,cols,i,fileName);
}else{
writePBM(thisBoard, rows, cols, fileName, i);
}
for (x = 0; x < cols; x++){
for (y = 0; y < rows; y++){
unsigned char * array = board_to_array(thisBoard, rows, cols);
unsigned int neighbours = vecinos(array, x, y, rows, cols);
free(array);
if (thisBoard[x][y]=='1'){
if (neighbours == 3 || neighbours == 2){nextBoard[x][y] = '1';}
}else{
if (neighbours == 3){nextBoard[x][y] = '1';}
}
}
}
copy_board(nextBoard,thisBoard,rows, cols);
free_board(nextBoard, rows, cols);
}
free_board(thisBoard, rows, cols);
}

void print_help(){
printf("Uso:\n");
printf("  conway -h\n  conway -V\n  conway i M N inputfile [-o outputprefix]\n");
printf("Opciones:\n");
printf("  -h, --help    Imprime este mensaje\n");
printf("  -V, --version Da la version del programa\n");
printf("  -o            Prefijo de los archivos de salida\n");
printf("Ejemplos: \n");
printf("  conway 10 20 20 glider -o estado\n");
printf("  Representa 10 iteraciones del Juego de la Vida en una Matriz\n");
printf("  de 20x20, con un estado inicial tomado del archivo ''glider''.\n");
printf("  Los archivos de salida se llamarán estado_n.pbm,\n");
printf("  si no se da un prefijo para el archivo de salida, \n");
printf("  el prefijo será el nombre del archivo de entrada.\n");
}

void print_version (){
printf("Conway -Game of Life- version: 1.0\n");
}

void validate_actionsCount(int actionsCount){
if (actionsCount <= 0){
fprintf(stderr, "Actions Count must be a positive integer!\n");
exit(1);
}
}

void validate_rows(int rows){
if (rows <= 0){
fprintf(stderr, "Number of rows must be a positive integer!\n");
exit(1);
}
}

void validate_cols(int cols){
if (cols <= 0){
fprintf(stderr, "Number of columns must be a positive integer!\n");
exit(1);
}
}

int main(int argc, char* argv[])
{
char const *fileName;
char const *outputFileName;
unsigned int actionsCount;
unsigned int rows;
unsigned int cols;
unsigned char **board;
bool screen = 0;

if (argc == 2){
char* arg = argv[1];
if (strcmp(arg,"-h") == 0 || strcmp(arg,"--help") == 0){
print_help();
return 0;
}
else if (strcmp(arg,"-v") == 0 || strcmp(arg,"--version") == 0){
print_version();
return 0;
}
else{
fprintf(stderr,"Invalid parameter!\n");
exit(1);
}
}else if (argc != 5 && argc != 7 && argc != 6){ //ver
fprintf(stderr,"Wrong number of parameters!\n");
exit(1);
}else{
fileName = argv[4];
actionsCount = (int) atoi(argv[1]);
validate_actionsCount(actionsCount);
rows = (int) atoi(argv[2]);
validate_rows(rows);
cols = (int) atoi(argv[3]);
validate_cols(cols);
outputFileName = fileName;
if (argc == 7){
if (strcmp(argv[5],"-o") == 0){
outputFileName = argv[6];
}else{
fprintf(stderr,"Invalid parameter!\n");
exit(1);
}
}
if (argc == 6){
if (strcmp(argv[5],"-s") == 0){
screen = 1;
}else{
fprintf(stderr,"Invalid parameter!\n");
exit(1);
}
}
}

FILE* file = fopen(fileName, "r");
if (file==NULL){
fprintf(stderr, "Error while opening input file\n");
exit(1);
}

board = init_board(rows, cols);
load_board(board, rows, cols, file);
fclose(file);
process_board(board, rows, cols, actionsCount, outputFileName, screen);
free_board(board, rows, cols);

return 0;
}
\end{lstlisting}

\subsection{Función Vecinos en Mips}

\begin{lstlisting}[caption={vecinos.s},label={lst:codigoc}]
include <mips/regdef.h>
#include <sys/syscall.h>

######################################################################################
# Empieza el codigo...
# los nombres de los registros estan mal es para ubicarlos facil y despues cambiarlos
# la cosa es asi las 8 comparaciones i +/- j +/- si pasa entra a la funcion donde se accede
# a i j de la matriz y verifica que sea igual a 1 devuelve 1 sino 0, falta allocar bien los stack
# de la principal y de la funcion
######################################################################################
.text
.align 2
.globl vecinos
.ent vecinos
vecinos:
subu	sp, sp, 76 #creo stack
sw	ra, 40(sp) #guardo el ra en stack
sw	$fp, 36(sp)	#guardo el fp en stack
sw	gp, 32(sp) #guardo el gp en stack
sw	s0, 44(sp) #guardo el s0 en stack
sw	s1, 48(sp) #guardo el s1 en stack
sw	s2, 52(sp) #guardo el s2 en stack
sw	s3, 56(sp) #guardo el s3 en stack
sw	s4, 60(sp) #guardo el s4 en stack
sw	s5, 64(sp) #guardo el s5 en stack
sw	s6, 68(sp) #guardo el s6 en stack
sw	s7, 72(sp) #guardo el s7 en stack
sw	a0, 0(sp) #guardo el 1er parametro N en el stack  en stack
sw	a1, 4(sp) #guardo el 2do parametro N en el stack  en stack
sw	a2, 8(sp) #guardo el 3er parametro N en el stack  en stack
sw	a3, 12(sp) #guardo el 4to parametro N en el stack  en stack
lw	t0, 92(sp) #cargo el 5to parametro N en el stack  en stack
sw	t0, 16(sp) #guardo el 5to parametro N en el stack  en stack
move	$fp, sp #muevo el fp al inicio del stack
move	t1, a0 #cargo el 1er parametro a* en t1 
move	t2, a1 #cargo el 2do parametro i en t2
move	t3, a2 #cargo el 3er parametro j en t3
move	t4, a3 #cargo el 4tp parametro M en t4
move	s1, t2 #i
move	s2, t3 #j
addi	s3, s1, 1 #imasuno
addi	s4, s2, 1 #jmasuno
addi	s5, s1, -1	#imenosuno
addi	s6, s2, -1 #jmenosuno
move	s0, t4 #m
move	s7, t0 #n
move	t8, t1 #matriz
li	t6, 0 #en t6 se va a ir acumulando el valor
blt	s3, s0, imas #compara si i + 1 < m 
comp2:
blt	s4, s7, jmas #compara si j + 1 < n 
comp3:
bge	s5, zero, imenos #compara si i - 1 >= 0 
comp4:
bge	s6, zero, jmenos #compara si j - 1 >= 0
comp5:
bge	s3, s0, comp6 #compara si i + 1 >= m
blt	s4, s7, imasjmas #compara si j + 1 < n
comp6:
bge	s3, s0, comp7 #compara si i + 1 >= m
bge	s6, zero, imasjmenos #compara si j - 1 >= 0 
comp7:
blt	s5, zero, comp8 #compara si i + 1 < 0
blt	s4, s7, imenosjmas #compara si j + 1 < n 
comp8:
blt	s5, zero, fin #compara si i + 1 < 0
bge	s6, zero, imenosjmenos #compara si j - 1 >= 0 
#falta desapilar stack
fin:
#move	v0, s7
lw	ra, 40(sp) #empiezo a liberar stack, cargo el ra del stack
lw	$fp, 36(sp) #cargo el fp del stack
sw	s7, 16(sp) #el param 5
lw	gp, 32(sp) #cargo el gp del stack
lw	s0, 44(sp) #cargo el s0 del stack
lw	s1, 48(sp) #cargo el s1 del stack
lw	s2, 52(sp) #cargo el s2 del stack
lw	s3, 56(sp) #cargo el s3 del stack
lw	s4, 60(sp) #cargo el s4 del stack
lw	s5, 64(sp) #cargo el s5 del stack
lw	s6, 68(sp) #cargo el s6 del stack
lw	s7, 72(sp) #cargo el s7 del stack
lw	a0, 0(sp) #cargo el param 1
lw	a1, 4(sp) #cargo el param 2
lw	a2, 8(sp) #cargo el param 3
lw	a3, 12(sp) #cargo el param 4

addiu	sp, sp, 76
move	v0, t6
jr	ra

imas: #se llama a a[i+1][j] para ver si esta vivo o no
move	a0, s3
move	a1, s2
move	a2, s7
move	a3, t8
jal	accessijpos
add	t6, v0, t6
b	comp2
jmas: #se llama a a[i][j+1] para ver si esta vivo o no
move	a0, s1
move	a1, s4
move	a2, s7
move	a3, t8
jal	accessijpos
add	t6, v0, t6
b	comp3
imenos: #se llama a a[i-1][j] para ver si esta vivo o no
move	a0, s5
move	a1, s2
move	a2, s7
move	a3, t8
jal accessijpos
add t6, v0, t6
b comp4
jmenos: #se llama a a[i][j-1] para ver si esta vivo o no
move	a0, s1
move	a1, s6
move	a2, s7
move	a3, t8
jal accessijpos
add t6, v0, t6
b comp5
imasjmas: #se llama a a[i+1][j+1] para ver si esta vivo o no
move	a0, s3
move	a1, s4
move	a2, s7
move	a3, t8
jal accessijpos
add t6, v0, t6
b comp6
imasjmenos: #se llama a a[i+1][j-1] para ver si esta vivo o no
move	a0, s3
move	a1, s6
move	a2, s7
move	a3, t8
jal accessijpos
add t6, v0, t6
b comp7
imenosjmas: #se llama a a[i-1][j+1] para ver si esta vivo o no
move	a0, s5
move	a1, s4
move	a2, s7
move	a3, t8
jal accessijpos
add t6, v0, t6
b comp8
imenosjmenos: #se llama a a[i-1][j-1] para ver si esta vivo o no
move	a0, s5
move	a1, s6
move	a2, s7
move	a3, t8
jal accessijpos
add t6, v0, t6
b fin

.globl accessijpos
accessijpos:
subu	sp, sp, 32 #creo stack
sw	ra,	24(sp) #guardo el ra en stack
sw	$fp, 20(sp)	#guardo el fp en stack
sw	gp, 16(sp) #guardo el gp en stack
move	t2, a0 # pos i
move	t4, a1 # pos j
move	t3, a2 # n de la matriz
move	t0, a3 # pos 0,0 de la matriz
mul	t5, t3, t2 # $t5 <-- width * i
add	t5, t5, t4 # $t5 <-- width * i + j
add	t5, t0, t5 # $t5 <-- base address + (2^2 * (width * i + j))
lbu	t1, 0(t5) #carga el '1' o '0' de la matriz
addiu	t1, t1,-48 # carga '1' en t7 para comparar
move	v0, t1
lw	ra, 24(sp) #empiezo a liberar stack, cargo el ra del stack
lw	$fp, 20(sp) #cargo el fp del stack
lw	gp, 16(sp) #cargo el gp del stack
addiu	sp, sp, 32
jr	ra
.end	vecinos
\end{lstlisting}

\subsection{Función Vecinos en C}

\begin{lstlisting}[caption={vecinos.c},label={lst:codigoc}]
unsigned int vecinos(unsigned char *a, unsigned int i, unsigned int j, unsigned int M, unsigned int N){
	unsigned int vecinos = 0;
	int x,y;
	for (x = -1; x<2; x++){
		for (y = -1; y<2; y++){
			if (x+i>=0 && x+i<N && y+j>=0 && y+j<M){
				if ((!(x==0&&y==0)) && a[N*(x+i) + (y+j)] == '1'){
					vecinos++;
				}
			}
		}
	}
	return vecinos;
}
\end{lstlisting}

\section{Dirección del Repositorio}

Todos los archivos correspondientes al trabajo practico se encuentran en este repositorio:

	https://github.com/fede29/orga\_tp1
	

\end{document}
